\chapter{Analyse de l'existant}

\section{ExaBGP}
ExaBGP est une application conçue pour permettre aux programmeurs et aux administrateur réseau d’interagir facilement avec les réseaux BGP. Le programme est conçu pour permettre l’injection de routes dans un réseau, y compris IPv6 et FlowSpec.
\cite{Man10}

\section{Erco}
Erco (Exabgp ROutes COntroller) est un logiciel permettant de piloter Exabgp. Il permet l'ajout et la suppression de routes. C'est un logiciel libre exposant une API REST, lui permettant ainsi d'être utilisé par des programmes.

\section{Django}
Django est un framework Python gratuit et open source consacré au développement web. Il propose des bases simples et déjà implémentées.

\section{Flask-RESTful}
Flask-RESTful est une extension du framework Flask permettant de construire des APIs REST.


\section{Méthode de virtualisation avec NEmu}
NEmu (Network Emulator for Mobile Universes) est un environnement de réseaux virtuels distribué écrit en pyhton. Il gère des machines virtuelles de type QEmu pour construire une topologie virtuelle dynamique.
Ainsi nous pourrions facilement créer un ensemble de machine pour figurer un réseau et simuler l'envoie de paquet d'une machine attaquante.

\section{Projet de l'an dernier}
