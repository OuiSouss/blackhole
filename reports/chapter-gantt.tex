\chapter{Diagramme de Gantt}

\begin{ganttchart}[
hgrid=true,
vgrid={{dotted}}
]{1}{18}
\gantttitle{Routage vers trou noir}{18} \\
\gantttitlelist{6,...,14}{2} \\
% red = geoff ; green = cyrielle ; blue = kévin ; yellow = amélie
\ganttbar[
bar/.append style={fill=red!40}
]{Connexion/Déconnexion}{1}{6} \\
\ganttbar[
bar/.append style={fill=green!40}
]{Créer routes}{5}{8} \\
\ganttbar[
bar/.append style={fill=green!40}
]{Supprimer routes}{9}{12} \\
\ganttbar[
bar/.append style={fill=blue!40}
]{Activer routes}{5}{8} \\
\ganttbar[
bar/.append style={fill=blue!40}
]{Désactiver routes}{9}{12} \\
\ganttbar[
bar/.append style={fill=yellow!40}
]{Envoyer routes à ExaBGP}{3}{8} \\
\ganttbar[
bar/.append style={fill=yellow!40}
]{Diffuser les routes}{9}{14} \\
\ganttbar[
bar/.append style={fill=red!40}
]{Lien UI/API}{7}{12} \\
\ganttbar[
bar/.append style={fill=blue!40}
]{Base de données routes}{1}{6} \\
\ganttbar[
bar/.append style={fill=green!40}
]{Base de données identifiants}{1}{6}  \\
\ganttbar[
bar/.append style={fill=red!40}
]{Générer cache}{13}{16} \\
\ganttbar[
bar/.append style={fill=green!40}
]{Conversion HTTP/JSON}{12}{15} \\
\ganttbar[
bar/.append style={fill=yellow!40}
]{Test Final}{15}{18}
\end{ganttchart}

\begin{tabular}{|c|c|c|c|c|}
\hline
Légende & \cellcolor{green!40} Cyrielle DUBOIS & \cellcolor{yellow!40} Amélie RISI & \cellcolor{red!40} Geoffroy GRESSIER & \cellcolor{blue!40} Kevin Maitre \\
\hline
\end{tabular}