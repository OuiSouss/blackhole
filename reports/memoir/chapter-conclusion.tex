\chapter*{Conclusion}
\addcontentsline{toc}{chapter}{Conclusion}

%Avons nous été bien organisé, les améliorations a apporté, ce que l'on pense des qualité de notre projet. 

%But
%Ce projet avait pour but de nous faire travailler sur un moyen de se protéger face à des attaques par dénis de service. Nous avons donc dévelop

%afin de le faciliter dans sa tache
%admin réseau

%appris
%Lors de ces semaines, nous avons appris l'utilisation de nouvelles technologie mais aussi l
% C : programmation d'une api flask restful avec gestion des request et reponse comme GET POST, utilisation d'exabgp, gestion de projet très professionnels et rigoureux, mettre en place un environnement virtuel, => ça me semble tout pour moi

% web : Django, AdminLTE
% tests unitaires, performances
% PDP c'est différent d'un stage et de projet-transdisciplinaire

% on travaillait avec des technologies exigeantes donc il fallait être attentif

%résultat


%limite
%Malheureusement, notre application aurait pu répondre encore mieux à la problématique mais par manque de temps, nous n'avons pas pu ajouter toutes les fonctionnalités rendant son utilisation optimale.


Ce projet avait pour but de nous faire développer une solution afin de faciliter la protection d'un réseau face à des attaques par déni de service pour un administrateur réseau.\newline

Nous avons pu nous familiariser avec de nouvelles technologies; telles que le Web avec Django et AdminLTE, les API Restful avec Flask et Mlab, les bases de données avec SQLite et MongoDB ou encore la virtualisation avec ExaBGP et Nemu. De la même manière, nous avons pu consolider nos connaissances dans certains domaines, comme Git ou Python.\newline

Certaines technologies que nous utilisions étaient exigeantes et nécessitaient plus d'attention que d'habitude, comme la mise en place et l'utilisation d'un environnement virtuel pour fonctionner.\newline

Mais ce projet nous a aussi permis d'apprendre de nouvelles méthodologies de travail et d'organisation, comme comprendre et traduire les besoins d'un client, faire régulièrement des réunions pour mettre au point l'avancement du projet et déléguer le travail afin de progresser plus efficacement en équipe.\newline


Notre application aurait pu répondre encore mieux à la problématique, mais par manque de temps, nous n'avons pas pu ajouter toutes les fonctionnalités rendant son utilisation optimale.
Après avoir passé plusieurs semaines sur ce projet, nous sommes très satisfait du résultat final et de cette expérience, qui est différente de celle apportée par le projet de communication trans-disciplinaire de l'an dernier.\newline



